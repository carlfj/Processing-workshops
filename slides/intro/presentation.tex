
\documentclass{beamer}

% \mode<presentation>
\usetheme{Luebeck} %Warsaw
% \usecolortheme{seahorse}
% \usecolortheme{lily}
\usecolortheme[RGB={51,255,0}
]{structure}

\usepackage[utf8]{inputenc}
\usepackage[danish]{babel}
\usepackage{graphicx}
\usepackage{url}

\title{Processing.org workshops\\Intro}
\author{Open Space Aarhus}
\date{\today}
\institute[Bryggervej 30]{Bryggervej 30, 8240 Århus N}

% logo
\pgfdeclareimage[height=1.3cm]{university-logo}{osaa_logo_neon_rgb}
\logo{\pgfuseimage{university-logo}}

\begin{document}

\begin{frame}[label=titlepage]
  \titlepage
\end{frame}

\begin{frame}
  \frametitle{Hvem er OSAA?}
  \begin{itemize}
  \item Idé - fælleslokaler til nørder og foreninger
  \item Bliv medlem:
    \begin{itemize}
    \item Adgang til lokalerne 24/7
    \item Det er det, foreningen er baseret på
    \end{itemize}

  \item Vær med i OSAA:
    \begin{itemize}
    \item Brug lokalerne til dine aktiviteter
    \item Deltag i eksisterende aktiviteter - se \url{http://osaa.dk}
    \item Mød op om tirsdagen og leg med Hack Århus
    \item Åbne bestyrelsesmøder hver torsdag efter ${T^3}$
    \end{itemize}
  \end{itemize}						
\end{frame}

\begin{frame}
  \frametitle{Hvad vil vi}
  
  \begin{block}{Have det sjovt}
    At programmere er sjovt - at kode grafik er endnu sjovere.
  \end{block}

  \begin{block}{Lære noget nyt}
    At lære er en kontakt sport.
  \end{block}
  
\end{frame}


\begin{frame}
  \frametitle{Hvad er processing.org?}
  \begin{itemize}
  \item   Et framework til grafik programmering
  \item   Henvender sig til ikke programmører
  \item   Gør ting nemmere, mindre boilerplate.
  \end{itemize}
\end{frame}


\begin{frame}
  \frametitle{Hvem vi henvender os til}
  \begin{block}{Et nyt publikum}
    Flere ny medlemmer 

    Nå ud til et nyt publikum

\end{block}
  
\end{frame}

\begin{frame}
  \frametitle{Hvad vi vil gå igennem}
  Fra "hello world" til foo

  Hvad er :
  \begin{itemize}
  \item En variable
  \item Et loop
  \item En funktion
  \item \ldots{}
  \item Interaktiv visualisering af internettet.
  \end{itemize}
  
  \begin{block}{Hvad kan du?}
    Vi starter fra bunden, men tager udgangspunkt i publikum
  \end{block}


\end{frame}


\begin{frame}
  \frametitle{Velkendt}
   Det er det sammen IDE som
   \begin{itemize}
   \item Arduino
   \item replicatorG
   \end{itemize}

   Det henvender sig til samme kreativ segment.

 \end{frame}
 
 \begin{frame}
   \frametitle{Der er community}
   
   \begin{itemize}
   \item Der er ekstra libs
   \item Der er hjælp
   \item Der er interesse
   \end{itemize}
   
 \end{frame}
 
 \begin{frame}
   \frametitle{Hvorfor ikke X}
   
   \begin{itemize}
   \item directX
   \item OpenGL
   \item SDL
   \item JOGL
   \item assembly
   \end{itemize}
   
   \begin{block}{Ligesom Arduino, bare til kode}
     Let at gå til. Måske ikke alle muligheder for at fedte detaljer, men det virker.
   \end{block}
   

   \end{frame}
   
   \begin{frame}
     \frametitle{eksempler paa ting}
     \begin{itemize}
     \item OpenProcessing
     \end{itemize}

   \end{frame}
   
   \begin{frame}
     \frametitle{kom selv eller Inviter dine venner}

     \begin{block}{Safeticke}
       Aftale med safeticket - nu med X tilmeldte
     \end{block}
     
   \end{frame}
   
   \begin{frame}
     \frametitle{Hvad kunne fremtiden blive}

   \end{frame}
   

\end{document}

