
\documentclass{beamer}

% \mode<presentation>
\usetheme{Luebeck} %Warsaw
% \usecolortheme{seahorse}
% \usecolortheme{lily}
\usecolortheme[RGB={51,255,0}
]{structure}

\usepackage[utf8]{inputenc}
\usepackage[danish]{babel}
\usepackage{graphicx}
\usepackage{url}
\usepackage[normalem]{ulem}

\graphicspath{{images/}}
\newcommand{\FIG}[2]{
  \begin{figure}[]
    \centering
    \includegraphics[width=0.95\textwidth,keepaspectratio]{#1}
    \caption{#2}
    \label{fig:#1}
  \end{figure}
}

\newcommand{\FIGMED}[2]{
  \begin{figure}[]
    \centering
    \includegraphics[width=0.75\textwidth,keepaspectratio]{#1}
    \caption{#2}
    \label{fig:#1}
  \end{figure}
}


\newcommand{\FIGSMALL}[2]{
  \begin{figure}[htbp]
    \centering
    \includegraphics[width=0.4\textwidth,keepaspectratio]{#1}
    \caption{#2}
    \label{fig:#1}
  \end{figure}
}


\title{Processing.org workshops\\Workshop 3}
\author{Open Space Aarhus}
\date{\today}
\institute[Bryggervej 30]{Bryggervej 30, 8240 Århus N}

% logo
\pgfdeclareimage[height=1.3cm]{university-logo}{osaa_logo_neon_rgb}
\logo{\pgfuseimage{university-logo}}

\begin{document}

\begin{frame}[label=titlepage]
  \titlepage
\end{frame}

\begin{frame}
  \frametitle{Dagens program}
  \begin{itemize}
  \item Introduktion
  \item Resume af sidste gang
  \item Kode
    \begin{itemize}
    \item Løkker
    \item Arrays
    \item Funktioner
    \end{itemize}

  \item Workshop
    \begin{itemize}
    \item Mange bolde på en gang!
    \end{itemize}

  \item \emph{Afslutning}
    
  \end{itemize}						
\end{frame}


\begin{frame}
  \frametitle{OpenProcessing.org}
  \begin{block}{Hvad har de flittige lavet}
    \begin{itemize}
    \item \url{http://www.openprocessing.org/classrooms/?classroomID=1075}
    \end{itemize}
  \end{block}
\end{frame}


\begin{frame}
  \frametitle{Introduktion}
  
  \begin{block}{Slides og processing filer}
    \url{http://poodle/processing}   
  \end{block}
  {\tiny Slides kan sikkert bruges til at kigge i eller kopiere fra.}
\end{frame}

\begin{frame}[fragile]
  \frametitle{Resume}
  
  \begin{itemize}
  \item Variable: \texttt{float boldX = 200;}\\
  \item Operationer: \texttt{boldX = boldX + deltaX;}\\
  \item Løkker:
    \begin{itemize}
    \item \texttt{while (betingelse) \{ ... \}}\\
    \item \texttt{for (start; betingelse; opdatering;) \{ ... \}}\\
    \end{itemize}
  \item Forgreninger: 
    \begin{itemize}
    \item \texttt{if (\emph{x > width}) \{ dx = -dx; \}}\\
    \item \texttt{if (betingelse) \{ ... \}}\\
    \item \texttt{if (betingelse) \{ ... \} else \{ ... \}}\\
    \end{itemize}
  \item Simpel fysik.
    \begin{itemize}
    \item acceleration = summen af kræfter delt med partiklens masse
    \item ny hastighed = gammel hastighed + acceleration
    \item ny positioin = gammel position + hastighed
    \end{itemize}
  \end{itemize}  
\end{frame}

\begin{frame}[fragile]
  \frametitle{Løkker}

  Lav en løkke som tegner 10 bolde ved siden af hinanden. Brug x som tæller
  \begin{block}{med while-løkken}  
\begin{verbatim}
int x = 0;
while (x < 10) {
  ellipse(20 + x * 40, 200, 30, 30);
  x += 1;
}
\end{verbatim}  
  \end{block}
  
  \begin{block}{med for-løkken}
\begin{verbatim}
for (int x = 0; x < 10; x += 1) {
  ellipse(20 + x * 40, 200, 30, 30);
}
\end{verbatim}  
  \end{block}
    
\end{frame}

\begin{frame}[fragile]
  \frametitle{Opgave: Et gitter af figurer}

  Lav to løkker indeni hinanden, så rækken bliver gentaget under hinanden. Brug x og y som tæller. 
  
  \begin{block}{Løkker i løkker}    
\begin{verbatim}
for (int y = 20; y < height; y += 40) {
  for (int x = 20; x < width; x += 40) {
    ellipse(x, y, 30, 30);
  }
}
\end{verbatim}  
  \end{block}
  
Prøv at bruge x og y eller random() til at styre farve eller størrelse. Brug evt forgreninger til at tegne forskellige figurer.
  
\end{frame}


\begin{frame}[fragile]
  \frametitle{Arrays}
  
  Et array er en opslagstabel. Man kan lave et array med et fast antal pladser.
  Derefter kan man skrive og læse værdier på de enkelte pladser i arrayet.
  
  \begin{itemize}
  \item deklaration: float[] boldX = new float[10];
  \item tildeling: boldX[5] = 100; 
  \item læsning: ellipse(boldX[5], boldY[5], 30, 30);
  \end{itemize}
  
\end{frame}


\begin{frame}[fragile]
  \frametitle{Øvelse til arrays}
  
  Skal vi have en lille øvelse med arrays?
  
\end{frame}

\begin{frame}[fragile]
  \frametitle{Arrays til mange bolde}
 
  \begin{itemize}
  \item float[] boldX = new float[10];
  \item float[] boldY = new float[10];
  \item float[] deltaX = new float[10];
  \item float[] deltaY = new float[10];
  \end{itemize}
  
\end{frame}

\begin{frame}[fragile]
  \frametitle{Arrays og Løkker}
  
  Arrays og løkker er som skabt til hinanden. Vi kan bruge en for-løkke til 
  at løbe igennem kuglerne og opdatere deres positioner.

\begin{verbatim}
int ANTAL = 10;

float[] kugleX = new float[ANTAL];
float[] kugleY = new float[ANTAL];
float[] deltaX = new float[ANTAL];
float[] deltaY = new float[ANTAL];

for (int i = 0; i < ANTAL; i++) {
  kugleX[i] += deltaX[i];
  kugleY[i] += deltaY[i];
}
\end{verbatim}
  
   
\end{frame}

\begin{frame}[fragile]
  \frametitle{Demo}
  
  Kanoneksemplet udvidet til at skyde med flere kanonkugler samtidigt.
  
  http://www.openprocessing.org/visuals/?visualID=47076
  
\end{frame}

\begin{frame}[fragile]
  \frametitle{Funktioner}
  
  {\tiny Små genbrugelige stumper kode. Også nyttig til at gøre koden mere overskuelig. Du har allerede brugt en masse funktioner fra processing. Du har også skrevet dine egne fx. setup() og draw(). Nu vil vi lave vore egne}  
\end{frame}

\begin{frame}[fragile]
  \frametitle{Funktioner}
  
  {\tiny funktioner til dagens opgave...?}  
\end{frame}


\begin{frame}
  \frametitle{Tak for i dag}

  \begin{itemize}
  \item Hvad syntes \emph{du} om i dag?
  \item Næste gang: ?
  \item $T^3$ i må meget gerne hjælpe med at rydde lokalet.
  \end{itemize}

  \begin{block}{Klasseværelset}
    \url{www.openprocessing.org/classrooms/?classroomID=1075}
  \end{block}

\end{frame}
\end{document}

