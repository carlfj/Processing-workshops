
\documentclass{beamer}

% \mode<presentation>
\usetheme{Luebeck} %Warsaw
% \usecolortheme{seahorse}
% \usecolortheme{lily}
\usecolortheme[RGB={51,255,0}
]{structure}

\usepackage[utf8]{inputenc}
\usepackage[danish]{babel}
\usepackage{graphicx}
\usepackage{url}
\usepackage[normalem]{ulem}

\graphicspath{{images/}}
\newcommand{\FIG}[2]{
  \begin{figure}[]
    \centering
    \includegraphics[width=0.95\textwidth,keepaspectratio]{#1}
    \caption{#2}
    \label{fig:#1}
  \end{figure}
}

\newcommand{\FIGMED}[2]{
  \begin{figure}[]
    \centering
    \includegraphics[width=0.75\textwidth,keepaspectratio]{#1}
    \caption{#2}
    \label{fig:#1}
  \end{figure}
}


\newcommand{\FIGSMALL}[2]{
  \begin{figure}[htbp]
    \centering
    \includegraphics[width=0.4\textwidth,keepaspectratio]{#1}
    \caption{#2}
    \label{fig:#1}
  \end{figure}
}


\title{Processing.org workshops\\Workshop 3}
\author{Open Space Aarhus}
\date{\today}
\institute[Bryggervej 30]{Bryggervej 30, 8240 Århus N}

% logo
\pgfdeclareimage[height=1.3cm]{university-logo}{osaa_logo_neon_rgb}
\logo{\pgfuseimage{university-logo}}

\begin{document}

\begin{frame}[label=titlepage]
  \titlepage
\end{frame}

\begin{frame}
  \frametitle{Dagens program}
  \begin{columns}
    \begin{column}{0.5\textwidth}
  
  \begin{itemize}
  \item Introduktion
  \item Resume af sidste gang
  \item Kode
    \begin{itemize}
    \item Arrays og løkker
    \item Funktioner
    \item Klasser
    \end{itemize}

  \item Grafik
    \begin{itemize}
    \item Mange bolde på en gang!
    \item Fyrværkeri (fontæne)
    \item Partikelsjov
    \end{itemize}

  \item \emph{Afslutning}
  \end{itemize}						

    \end{column}
    
    \begin{column}{0.5\textwidth}
      \FIG{dagensprogram}{Workshop 3}
    \end{column}
  \end{columns}
    
\end{frame}


\begin{frame}
  \frametitle{Introduktion}
  
  \begin{block}{Slides og processing filer}
    \url{http://poodle/processing}   
  \end{block}
  {\tiny Slides kan sikkert bruges til at kigge i eller kopiere fra.}
\end{frame}


\begin{frame}
  \frametitle{OpenProcessing.org}
  \begin{block}{Hvad har de flittige lavet}
    \begin{itemize}
    \item \url{http://www.openprocessing.org/classrooms/?classroomID=1075}
    \end{itemize}
  \end{block}
\end{frame}



\begin{frame}[fragile]
  \frametitle{Resume af sidste gang}
  
  \begin{itemize}
  \item Variable: \texttt{float boldX = 200;}\\
  \item Operationer: \texttt{boldX = boldX + deltaX;}\\
  \item Løkker:
    \begin{itemize}
    \item \texttt{while (betingelse) \{ ... \}}\\
    \item \texttt{for (start; betingelse; opdatering;) \{ ... \}}\\
    \end{itemize}
  \item Forgreninger: 
    \begin{itemize}
    \item \texttt{if (betingelse) \{ ... \}}\\
    \item \texttt{if (betingelse) \{ ... \} else \{ ... \}}\\
    \item \texttt{if (\emph{x > width}) \{ dx = -dx; \}}\\
    \end{itemize}
  \item Simpel fysik.
    \begin{itemize}
    \item acceleration = summen af kræfter delt med partiklens masse
    \item ny hastighed = gammel hastighed + acceleration
    \item ny position = gammel position + hastighed
    \end{itemize}
  \item Spørgsmål?
  \end{itemize}  
\end{frame}


\begin{frame}[fragile]
  \frametitle{More, More, MOOOOOOOOOREEE!!!}
  
  Vi vil have flere bolde på skærmen på samme tid!
  
  \begin{columns}
    \begin{column}{0.5\textwidth}
      \begin{block}{Erklæring}
\begin{verbatim}
float bold1X;
float bold1Y;
float delta1X;
float delta1Y;

float bold2X;
float bold2Y;
float delta2X;
float delta2Y;
\end{verbatim}
      \end{block}
    \end{column}
    
    \begin{column}{0.5\textwidth}
      \begin{block}{Opdatering}
\begin{verbatim}
bold1X += delta1X;
bold1Y += delta1Y;

bold2X += delta2X;
bold2Y += delta2Y;

if (bold1X > width) ...

if (bold2X > width) ...
\end{verbatim}
      \end{block}
    \end{column}
  \end{columns}
  
  Hvad så med 100 bolde? Der må være en nemmere måde...
\end{frame}


\begin{frame}[fragile]
  \frametitle{Arrays}
  
  Et array er en opslagstabel. Man kan lave et array med et fast antal pladser.
  Derefter kan man skrive og læse værdier på de enkelte pladser i arrayet.
  
  \begin{block}{Syntax}
  \begin{itemize}
  \item erklæring: type[] navn;
  \item initialisering: float[] boldX = new float[10];
  \item tildeling: boldX[5] = 100; 
  \item læsning: ellipse(boldX[5], boldY[5], 30, 30);
  \end{itemize}
  \end{block}
  
  Bemærk at det første element i et array har indeks 0. Det vil fx sige at et array med 10 elementer indekseres med tallene 0-9.
\end{frame}


\begin{frame}[fragile]
  \frametitle{Arrays til mange bolde}
  
  Hvis vi laver variablene for bolden om til arrays, kan vi holde styr
  på tilstanden af mange bolde.
 
  \begin{block}{Boldenes tilstandsvariable som arrays}
\begin{verbatim}
float[] boldX = new float[10];
float[] boldY = new float[10];

float[] deltaX = new float[10];
float[] deltaY = new float[10];
\end{verbatim}
  \end{block}
  
  Nu kan vi bruge løkker til at opdatere og tegne boldene.
  
\end{frame}


\begin{frame}[fragile]
  \frametitle{To måder at lave en løkke med tæller-variabel på}

  Lav en løkke som tegner 10 bolde ved siden af hinanden. Brug x som tæller
  \begin{block}{med while-løkken}  
\begin{verbatim}
int x = 0;
while (x < 10) {
  ellipse(20 + x * 40, 200, 30, 30);
  x += 1;
}
\end{verbatim}  
  \end{block}
  
  \begin{block}{med for-løkken}
\begin{verbatim}
for (int x = 0; x < 10; x += 1) {
  ellipse(20 + x * 40, 200, 30, 30);
}
\end{verbatim}  
  \end{block}
    
\end{frame}

\begin{frame}[fragile]
  \frametitle{Et gitter af figurer}

  Lav to løkker indeni hinanden, så rækken bliver gentaget under hinanden. Brug x og y som tæller. 
  
  \begin{block}{Løkker i løkker}    
\begin{verbatim}
for (int y = 20; y < height; y += 40) {
  for (int x = 20; x < width; x += 40) {
    ellipse(x, y, 30, 30);
  }
}
\end{verbatim}  
  \end{block}
  
"Hjemmeopgave": Prøv at bruge x og y eller random() til at styre farve eller størrelse. Brug evt forgreninger til at tegne forskellige figurer.
  
\end{frame}


\begin{frame}[fragile]
  \frametitle{Arrays og Løkker}
  
  Arrays og løkker er som skabt til hinanden. Vi kan bruge en for-løkke til 
  at løbe igennem alle kuglerne og opdatere deres positioner.

\begin{verbatim}
int ANTAL = 10;

float[] kugleX = new float[ANTAL];
float[] kugleY = new float[ANTAL];
float[] deltaX = new float[ANTAL];
float[] deltaY = new float[ANTAL];

for (int i = 0; i < ANTAL; i++) {
  kugleX[i] += deltaX[i];
  kugleY[i] += deltaY[i];
}
\end{verbatim}
  
   
\end{frame}

\begin{frame}[fragile]
  \frametitle{Så skal der kodes!}
  
  \begin{block}{Udgangspunktet}
  Kanoneksemplet fra sidste gang
  
  http://www.openprocessing.org/visuals/?visualID=46710
  \end{block}
    
\end{frame}


\begin{frame}[fragile]
  \frametitle{Så skal der kodes!}
      
  \begin{block}{Resultatet}
  Kanoneksemplet udvidet til at skyde med flere kanonkugler samtidigt.
  
  http://www.openprocessing.org/visuals/?visualID=47076
  \end{block}

\end{frame}


\begin{frame}[fragile]
  \frametitle{Funktioner}
  
  Små genbrugelige stumper kode. Også nyttig til at gøre koden mere overskuelig.\\
  \vspace{0.5cm}
  Du har allerede brugt en masse funktioner fra processing. Du har også skrevet dine egne fx. setup() og draw(). Nu vil vi lave vore helt egne.
  
  \begin{block}{En funktion:} 
  \begin{itemize}
  \item kan tage input via parametre
  \item kan returnere et output
  \item kan have sideeffekter (f.eks. tegne på skærmen)
  \end{itemize}
  \end{block}

\end{frame}


\begin{frame}[fragile]
  \frametitle{Funktioner}
    
\begin{verbatim}
returtype navn(parametre) {
 // implementation
}
\end{verbatim}
  
  \begin{block}{Eksempler}
  {\tiny
\begin{verbatim}
void draw() {
}

void drawTarget(float x, float y) {
  fill(255, 0, 0);
  ellipse(x, y, 40, 40);
  fill(255, 255, 255);  
  ellipse(x, y, 30, 30);
  fill(255, 0, 0);
  ellipse(x, y, 10, 10);  
}

float vinkelTilMus(float x, float y) {
  return atan2(mouseY - y, mouseX - x);
}

\end{verbatim}    
}
  \end{block}

\end{frame}



\begin{frame}[fragile]
  \frametitle{Fra kanon til fontæne}
  
  \begin{block}{Slagplan}
    \begin{itemize}
    \item skyd automatisk
    \item mindre kugler
    \item flere kugler
    \item hold styr på kuglernes alder
    \item ændr kuglernes farve baseret på deres alder
    \end{itemize}
  \end{block}
    
\end{frame}

\begin{frame}[fragile]
  \frametitle{Så skal der kodes!}
      
  \begin{block}{Resultatet}  
  http://www.openprocessing.org/visuals/?visualID=47191
  \end{block}

\end{frame}



\begin{frame}
  \frametitle{Klasser}

  \begin{itemize}
  \item Samler variable og de funktioner der benytter variablene i en logisk enhed
  \item Genbrugelig
  \item Masser af nyttige klasser på nettet
  \item Man behøver ikke forstå implementationen for at bruge dem! 
  \item Abstraktion
  \item Objektorienteret programmering
  \end{itemize}

\end{frame}


\begin{frame}
  \frametitle{Tak for i dag}

  \begin{itemize}
  \item Hvad syntes \emph{du} om i dag?
  \item Næste gang: ?
  \item $T^3$ på tirsdag. I må meget gerne hjælpe med at rydde lokalet.
  \end{itemize}

  \begin{block}{Klasseværelset}
    \url{www.openprocessing.org/classrooms/?classroomID=1075}
  \end{block}

\end{frame}
\end{document}

