
\documentclass{beamer}

% \mode<presentation>
\usetheme{Luebeck} %Warsaw
% \usecolortheme{seahorse}
% \usecolortheme{lily}
\usecolortheme[RGB={51,255,0}
]{structure}

\usepackage[utf8]{inputenc}
\usepackage[danish]{babel}
\usepackage{graphicx}
\usepackage{url}

\graphicspath{{images/}}
\newcommand{\FIG}[2]{
  \begin{figure}[]
    \centering
    \includegraphics[width=0.95\textwidth,keepaspectratio]{#1}
    \caption{#2}
    \label{fig:#1}
  \end{figure}
}

\newcommand{\FIGSMALL}[2]{
  \begin{figure}[htbp]
    \centering
    \includegraphics[width=0.4\textwidth,keepaspectratio]{#1}
    \caption{#2}
    \label{fig:#1}
  \end{figure}
}


\title{Processing.org workshops\\Intro}
\author{Open Space Aarhus}
\date{\today}
\institute[Bryggervej 30]{Bryggervej 30, 8240 Århus N}

% logo
\pgfdeclareimage[height=1.3cm]{university-logo}{osaa_logo_neon_rgb}
\logo{\pgfuseimage{university-logo}}

\begin{document}

\begin{frame}[label=titlepage]
  \titlepage
\end{frame}

\begin{frame}
  \frametitle{Dagens program}
  \begin{itemize}
  \item \url{http://openprocessing.org}
  \item Hvad er et partikelsystem
  \item Opsumering variable
  \item Kode
    \begin{itemize}
    \item Forgreninger
    \item Løkker
    \item Klasser
    \end{itemize}

  \item \emph{Afslutning}
    
  \end{itemize}						
\end{frame}


\begin{frame}
  \frametitle{OpenProcessing.org}
  \begin{block}{Hvad har de flittige lavet}
    \begin{itemize}
    \item \url{link 1}
    \item \url{link 2}
    \end{itemize}
  \end{block}
\end{frame}



\begin{frame}
  \frametitle{Partikelsystem}
  
  \emph{video der viser det er mere end bolde}

\end{frame}


\begin{frame}
  \frametitle{Teoretiske deltaljer}
  
  Først tager vi lige en hurtig og tildels teoretisk gennemgang af
  centrale begreber fra sidst + det nye vi skal bruge i dag. \\
\vspace{2cm}
  Derefter skal vi har \emph{beskidte fingre}.
\end{frame}


\begin{frame}
  \frametitle{Variable - \emph{erklæringer}}
  
  Der er indbyggede variable som
  \begin{itemize}
  \item mouseX
  \item mouseY
  \item width
  \end{itemize}

  Du kan \emph{erklære} din egne variable
  \begin{itemize}
  \item int x;
  \item float y;
  \end{itemize}

  \begin{block}{Erklæring}
    datatype navn;
  \end{block}

\end{frame}

\begin{frame}
  \frametitle{Variable - \emph{tildelinger}}
  
  Du kan \emph{tildele} en værdi til variable.
  \begin{itemize}
  \item x = 42;
  \item y = 3.14;
  \end{itemize}


  \begin{block}{OBS datatyper}
    int x; \\
    x = 3.14 \\
    x er nu 3, fordi den automagisk laver den om til et heltal.
  \end{block}

\end{frame}



\begin{frame}[fragile]
  \frametitle{Datatyper}
  \begin{columns}
    \begin{column}{0.4\textwidth}
      \begin{block}{int}
        Heltal (integer) : $1,2,42$
      \end{block}
      \begin{block}{float}
        komma tal: $3.14 \times 10^{42}$
      \end{block}

    \end{column}
    \begin{column}{0.6\textwidth}    
\begin{verbatim}
int x = 42;

float y = 3.14;
\end{verbatim}

    \end{column}
  \end{columns}  
\end{frame}

\begin{frame}[fragile]
  \frametitle{Datatyper - lidt andre}
  \begin{columns}
    \begin{column}{0.4\textwidth}
      \begin{block}{double}
        ligsom \texttt{float}, bare flere decimaler
      \end{block}
      \begin{block}{String}
        Tekst stykker: ``I'm a string''
      \end{block}
    \end{column}
    \begin{column}{0.6\textwidth}    
\begin{verbatim}

double z = 8.92838429338;


String hello = "world";

\end{verbatim}

    \end{column}
  \end{columns}  
\end{frame}



\begin{frame}[fragile]
  \frametitle{Boolean - \emph{endnu en datatype}}
  \begin{columns}
    \begin{column}{0.4\textwidth}
      \begin{block}{Boolean}
        Kan være \emph{sand} eller \emph{falsk}
      \end{block}
    \end{column}
    \begin{column}{0.6\textwidth}    
      \texttt{\emph{boolean} nemt = true; }\\
      \texttt{\emph{boolean} justinRocks = false; }
    \end{column}
  \end{columns}  
\end{frame}


\begin{frame}[fragile]
  \frametitle{Forgreninger}
  \begin{columns}
    \begin{column}{0.5\textwidth}
      \begin{block}{Sammenligninger}
        \begin{table}[h]
          \centering
          \begin{tabular}{lr}
            \hline
            Lighed & $==$ \\
            Ikke ens & $!=$\\
            Større end & $> $\\
            Mindre end & $<$\\
            Større end eller lig & $>=$\\
            Ditto for mindre & $<=$\\
            
          \end{tabular}
        \end{table}
      \end{block}
    \end{column}
    \begin{column}{0.5\textwidth}
\begin{verbatim} 
int x = 42;
int y = 0;

boolean foo = (x == y);
//false

boolean bar = (x >= y);
//true

\end{verbatim}
    \end{column}
  \end{columns}
\end{frame}


\begin{frame}[fragile]
  \frametitle{Boolske udtryke}
  \begin{columns}
    \begin{column}{0.5\textwidth}
      \begin{block}{Sammenligninger}
        \begin{table}[h]
          \centering
          \begin{tabular}{llr}
            Dansk & teknisk & kode \\
            \hline
            og & \texttt{and} & $\&\&$ \\
            enten & \texttt{or} & $|| $\\
            Ikke & \texttt{not} & $!$\\
            enten-eller & \texttt{xor} & $\wedge$\\
          \end{tabular}
        \end{table}
      \end{block}
    \end{column}
    \begin{column}{0.5\textwidth}
\begin{verbatim} 
boolean glad = true;
boolean sur = false;

boolean meh = (glad || sur);
boolean godEksempel = !sur;

\end{verbatim}
    \end{column}
  \end{columns}
\end{frame}



\begin{frame}
  \frametitle{Træning}

  \FIG{codingbat}{Eksempel fra \url{http://codingbat.com/java/Warmup-1}}

\end{frame}




\begin{frame}[fragile]
  \frametitle{Forgreninger}
  \begin{columns}
    \begin{column}{0.5\textwidth}
      \begin{block}{If-else-blokke}
        \texttt{if (\emph{boolean}) \{ }\\
        \texttt{// do stuff} \\
        \texttt{\} else \{ } \\
        \texttt{// do something else} \\
        \texttt{\}}\\
        \vspace{1cm}
      \end{block}
    \end{column}
    \begin{column}{0.5\textwidth}
\begin{verbatim} 
if (x < 200) {
  fill(255, 0, 0);
}

// if-else block
if (x < 200) {
  fill(255, 0, 0);
} else if (x < 300) {
  fill(0, 255, 0);
} else {
  fill(0, 0, 255);
}
\end{verbatim}
    \end{column}
  \end{columns}
\end{frame}


\begin{frame}[fragile]
  \frametitle{Løkker}
  \begin{columns}
    \begin{column}{0.5\textwidth}
      \begin{block}{looping}
        \texttt{while (\emph{boolean}) \{ }\\
        \texttt{// keep doing stuff} \\
        \texttt{\}}\\
        \vspace{1cm}
      \end{block}
    \end{column}
    \begin{column}{0.5\textwidth}
\begin{verbatim} 
int x = 0;
while (x < width) {
  point(x, 100);
  x++;
}
\end{verbatim}
    \end{column}
  \end{columns}
\end{frame}


\begin{frame}
  \frametitle{For loop \emph{ syntaktisk sukker}}

  der mangler stadigt noget , men det begynder at ligne noget
\end{frame}



\end{document}

